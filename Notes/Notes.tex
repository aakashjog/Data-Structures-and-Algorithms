\documentclass[titlepage, fleqn, a4paper, 12pt, twoside]{article}
\usepackage{etex}
\usepackage{geometry}
\usepackage{exsheets} %question and solution environments
\usepackage{amsmath, amssymb, amsthm} %standard AMS packages
\usepackage{esint} %integral signs
\usepackage{marginnote} %marginnotes
\usepackage{gensymb} %miscellaneous symbols
\usepackage{commath} %differential symbols
\usepackage{xcolor} %colours
\usepackage{cancel} %cancelling terms
\usepackage[free-standing-units]{siunitx} %formatting units
\usepackage{tikz, pgfplots} %diagrams
	\usetikzlibrary{calc, hobby, patterns, intersections, angles, quotes, spy}
\usepackage{graphicx} %inserting graphics
\usepackage{epstopdf} %converting and inserting eps graphics
\usepackage{hyperref} %hyperlinks
\usepackage{datetime} %date and time
\usepackage{enumerate, enumitem} %numbered lists
\usepackage{float} %inserting floats
\usepackage{microtype} %micro-typography
\usepackage{listings} %formatting code
	\lstset{language=Matlab}
	\lstdefinestyle{standardMatlab}
	{
		belowcaptionskip=1\baselineskip,
		breaklines=true,
		frame=L,
		xleftmargin=\parindent,
		language=C,
		showstringspaces=false,
		basicstyle=\footnotesize\ttfamily,
		keywordstyle=\bfseries\color{green!40!black},
		commentstyle=\itshape\color{purple!40!black},
		identifierstyle=\color{blue},
		stringstyle=\color{orange},
	}
\usepackage{algpseudocode} %algorithms
\usepackage{algorithm} %algorithms
\usepackage{todonotes}
\usepackage{booktabs}
\usepackage{bytefield}
\usepackage{xspace}
\usepackage{cleveref}

\newcommand\numberthis{\addtocounter{equation}{1}\tag{\theequation}} %adds numbers to specific equations in non-numbered list of equations

\theoremstyle{definition}
\newtheorem{example}{Example}
\newtheorem{definition}{Definition}

\theoremstyle{theorem}
\newtheorem{theorem}{Theorem}
\newtheorem{law}{Law}


\renewcommand{\algorithmicrequire}{\textbf{Input:}}
\renewcommand{\algorithmicensure}{\textbf{Output:}}

\let\And\relax
\algnewcommand\And{\ \textbf{and}\ }
\algnewcommand\False{\texttt{FALSE}}
\algnewcommand\True{\texttt{TRUE}}

\makeatletter
\@addtoreset{section}{part} %resets section numbers in new part
\makeatother

\newcommand\blfootnote[1]{%
	\begingroup
	\renewcommand\thefootnote{}\footnote{#1}%
	\addtocounter{footnote}{-1}%
	\endgroup
}

\renewcommand{\marginfont}{\scriptsize \color{blue}}

\renewcommand{\tilde}{\widetilde}

\SetupExSheets{solution/print = true} %prints all solutions by default

\renewcommand{\abstractname}{}    % clear the title

%opening
\title{Data Structures and Algorithms}
\author{Aakash Jog}
\date{2015-16}

\begin{document}

\pagenumbering{roman}
\begin{titlepage}
\newgeometry{margin=0cm}
\maketitle
\end{titlepage}
\restoregeometry
%\setlength{\mathindent}{0pt}

\blfootnote
{	
	\begin{figure}[H]
		\includegraphics[height = 12pt]{cc.pdf}
		\includegraphics[height = 12pt]{by.pdf}
		\includegraphics[height = 12pt]{nc.pdf}
		\includegraphics[height = 12pt]{sa.pdf}
	\end{figure}
	\noindent
	This work is licensed under the Creative Commons Attribution-NonCommercial-ShareAlike 4.0 International License. To view a copy of this license, visit \url{http://creativecommons.org/licenses/by-nc-sa/4.0/}.
} %CC-BY-NC-SA license

\tableofcontents

\clearpage
\section{Lecturer Information}

\textbf{Liron David}\\
~\\
Office: Tochna 313\\
E-mail: \href{mailto:lirondavid@gmail.com}{lirondavid@gmail.com}\\

\section{Required Reading}

\begin{enumerate}
	\item Introduction to Algorithms. Corman, Leiserson and Rivest (CLR)
\end{enumerate}

\section{Additional Reading}

\begin{enumerate}
	\item Data Structures and Algorithms. Aho, Hopcroft and Ullman (AHU)
\end{enumerate}

\clearpage
\pagenumbering{arabic}

\part{Introduction}

\section{Asymptotic Notation}

\begin{definition}[$\mathrm{O}$ notation]
	Let $f,g : \mathbb{N} \to \mathbb{R}^+$ be two functions.
	Then, $f(n)$ is said to be $\mathrm{O}$ of $g(n)$, if there exists a constant $c \in \mathbb{R}^+$, and there exists $n_0 \in \mathbb{N}$, such that for all $n \ge n_0$,
	\begin{align*}
		f(n) & \le c g(n)
	\end{align*}
	It is denoted as
	\begin{align*}
		f(n) & = \mathrm{O}\left( g(n) \right)
	\end{align*}
	This notation gives an asymptotic upper bound of $f(n)$.
\end{definition}

\begin{definition}[$\mathrm{\Omega}$ notation]
	Let $f,g : \mathbb{N} \to \mathbb{R}^+$ be two functions.
	Then, $f(n)$ is said to be $\mathrm{\Omega}$ of $g(n)$, if there exists a constant $c \in \mathbb{R}^+$, and there exists $n_0 \in \mathbb{N}$, such that for all $n \ge n_0$,
	\begin{align*}
		f(n) & \ge c g(n)
	\end{align*}
	It is denoted as
	\begin{align*}
		f(n) & = \mathrm{\Omega}\left( g(n) \right)
	\end{align*}
	This notation gives an asymptotic lower bound of $f(n)$.
\end{definition}

\begin{definition}[$\mathrm{\Theta}$ notation]
	Let $f,g : \mathbb{N} \to \mathbb{R}^+$ be two functions.
	Then, $f(n)$ is said to be $\mathrm{\Theta}$ of $g(n)$, if there exist constants $c_1,c_2 \in \mathbb{R}^+$, and there exists $n_0 \in \mathbb{N}$, such that for all $n \ge n_0$,
	\begin{equation*}
		c_1 g(n) \le f(n) \le c_2 g(n)
	\end{equation*}
	It is denoted as
	\begin{align*}
		f(n) & = \mathrm{\Theta}\left( g(n) \right)
	\end{align*}
	This notation gives both upper and lower bounds of $f(n)$.
\end{definition}

\section{Linear Search}

\begin{definition}[Array]
	A fixed number of elements of the same type, stored sequentially in memory, is called an array.
	The first element is assigned an index of 1.
	\begin{figure}[H]
		\begin{bytefield}[bitwidth = 3em, lsb=1]{9}
			\bitheader{1-9}\\
			\bitboxes{1}{{-23} {97} {18} {21} {5} {-86} {64} {0} {-37}}
		\end{bytefield}
	\end{figure}
\end{definition}

\begin{algorithm}[H]
	\begin{algorithmic}[1]
		\Require{Integer $n$, array $A$ of $n$ integers, integer $x$}
		\Ensure{Index of $x$}
		\Statex
		\State{$\texttt{found} \gets \False$}
		\State{$i \gets 1$}
		\While{$\texttt{found} = \False \And i \le n$}
			\If{$A[i] = x$}
				\State{$\texttt{found} \gets \True$}
			\Else
				\State{$i \gets i + 1$}
			\EndIf
		\EndWhile
		\If{$\texttt{found} = \True$}
			\State{\Return{$i$}}
		\Else
			\State{\Return{\texttt{NOT-FOUND}}}
		\EndIf
	\end{algorithmic}
	\caption{Linear Search}
	\label{alg:Linear_Search}
\end{algorithm}

\subsection{Correctness}

If $x$ exists in $A$, then there exists an index $i$, such that
\begin{align*}
	A[i] & = x
\end{align*}
In this case, the algorithm returns the index $i$.\\
If $x$ does not exist in $A$, then the algorithm returns \texttt{NOT-FOUND}.\\
Therefore, the algorithm works as required.

\subsection{Running Time}

Let $c_1$ be the number of operations executed before and after the while loop.
Therefore, $c_1$ is constant and independent of $n$.\\
Let $c_2$ be the number of operations executed in each iteration of the while loop.
Therefore, $c_2$ is constant.\\
Let $r$ be the number of iterations of the while loop.\\
Therefore, the total number of operations is
\begin{align*}
	T & = c_1 + c_2 r
\end{align*}
Therefore, as $r \le n$,
\begin{align*}
	T & \le c_1 + c_2 n
\end{align*}
Therefore, the running time is bounded from above by a linear function of $n$.
Therefore,
\begin{align*}
	T & = \mathrm{O}(n)
\end{align*}

\section{Binary Search}

\begin{algorithm}[H]
	\begin{algorithmic}[1]
		\Require{Integer $n$, sorted array $A$ of $n$ integers, integer $x$}
		\Ensure{Index of $x$}
		\Statex
		\Function{Binary Search}{$n$, $A$, $x$}
		\State{$\texttt{min} \gets 1$}
		\State{$\texttt{max} \gets n$}
		\State{$\texttt{found} \gets \False$}
		\While{$\texttt{found} = \False \And \texttt{min} \le \texttt{max} $}
		\State{$\texttt{mid} \gets \left\lfloor \frac{\texttt{min} + \texttt{max}}{2} \right\rfloor$}
			\If{$A[\texttt{mid}] = x$}
				\State{$\texttt{found} \gets \True$}
			\ElsIf{$x < A[\texttt{mid}]$}
				\State{$\texttt{max} \gets \texttt{mid} - 1$}
			\Else
				\State{$\texttt{min} \gets \texttt{mid} + 1$}
			\EndIf
		\EndWhile
		\If{$\texttt{found} = \True$}
			\State{\Return{\texttt{mid}}}
		\Else
			\State{\Return{\texttt{NOT-FOUND}}}
		\EndIf
		\EndFunction
	\end{algorithmic}
	\caption{Binary Search}
	\label{alg:Binary_Search}
\end{algorithm}

\subsection{Correctness}

Is $x$ exists in $A$, then by the Law of Trichotomy, $x$ is either less than, equal to, or greater than $A[\texttt{mid}]$.\\
If $x < A[\texttt{mid}]$, then $x$ must appear in the left half of the array which is currently being handled, i.e. $x$ must be in $A[\texttt{min},\dots,\texttt{mid} - 1]$.
This case is handled by setting the new array to be the left half, i.e. setting \texttt{max} to be $\texttt{mid} - 1$, and repeating the same procedure.\\
If $x = A[\texttt{mid}]$, then the first if condition is fulfilled, and the index is returned.
If $x > A[\texttt{mid}]$, then $x$ must appear in the right half of the array which is currently being handled, i.e. $x$ must be in $A[\texttt{mid} + 1,\dots,\texttt{min}]$.
This case is handled by setting the new array to be the right half, i.e. setting \texttt{min} to be $\texttt{mid} + 1$, and repeating the same procedure.\\
~\\
If $x$ does not exist in $A$, then the value of \texttt{found} is never changed, and remains \False.
The loop is terminated when $\texttt{min} > \texttt{max}$, and \texttt{NOT-FOUND} is returned.\\
Therefore, the algorithm works as required.

\subsection{Running Time}

Let $c_1$ be the number of operations executed before and after the while loop.
Therefore, $c_1$ is constant and independent of $n$.\\
Let $c_2$ be the number of operations executed in each iteration of the while loop.
Therefore, $c_2$ is constant.\\
Let $r$ be the number of iterations of the while loop.\\
Therefore, the total number of operations is
\begin{align*}
	c & = c_1 + c_2 r
\end{align*}
For each iteration of the while loop, the length of the array being used is halved.
Therefore, the maximum number of iterations is such that
\begin{align*}
	2^x          & \le n \\
	\therefore x & \le \log_2 n
\end{align*}
Therefore, the maximum number of iterations is $\lfloor\log_2 n\rfloor$.
\begin{align*}
	c & \le c_1 + c_2 \log_2 n
\end{align*}
Therefore, the running time is bounded from above by a logarithmic function of $n$.
Therefore,
\begin{align*}
	c & = \mathrm{O}(\log_2 n)
\end{align*}

\begin{question}
	Let $A$ be an array of size $n$, which consists of $0$s and $1$s only, and all $0$s appear before the $1$s.\\
	Write a pseudocode which receives such an array, and outputs the index of the switch from $0$ to $1$, i.e. it outputs the index $i$ such that $\forall j \le i$,
	\begin{align*}
		A[j] & = 0
	\end{align*}
	and $\forall j > i$,
	\begin{align*}
		A[j] & = 1
	\end{align*}
	Hint: Think of an algorithm that executes $c_1 + c_2 \log n$ operations.
\end{question}

\begin{solution}
	\begin{algorithm}[H]
		\begin{algorithmic}[1]
			\Require{Array $A$ of size $n$}
			\Ensure{Position of switch from $0$s to $1$s}
			\Statex
			\Function{Find Switch}{$A$, $n$}
				\If{$A[1] = 1$}
					\State{\Return{$0$}}
				\EndIf
				\If{$A[n] = 0$}
					\State{\Return{$n$}}
				\EndIf
				\State{$\texttt{min} \gets 1$}
				\State{$\texttt{max} \gets n$}
				\While{$\texttt{min} < \texttt{max} - 1$}
					\State{$\texttt{mid} \gets \left\lfloor \frac{\texttt{min} + \texttt{max}}{2} \right\rfloor$}
					\If{$A[\texttt{mid}] = 1$}
						\State{$\texttt{max} \gets \texttt{mid}$}
					\Else
						\State{$\texttt{min} \gets \texttt{mid}$}
					\EndIf
				\EndWhile
				\State{\Return{\texttt{min}}}
			\EndFunction
		\end{algorithmic}
		\caption{Finding switch from $0$s to $1$s in an sorted array of $0$s and $1$s only}
	\end{algorithm}
\end{solution}

\begin{question}
	Let $A$ be an array of size $n$, which consists of $0$s, and a single $1$.\\
	Assume that direct access to the array is not possible, i.e. using $A[i]$ to access the $i$th element of the array is not possible.
	Let \textsc{Sum}($A$, $i$, $j$) be a function which returns $\sum\limits_{k = i}^{j} A[k]$.\\
	Write a pseudocode that uses \textsc{Sum} and returns the index $i$ for which $A[i] = 1$.
	The function \textsc{Sum} should be called at most $\lceil \log n \rceil$ times.\\
	Hint: Think of \nameref{alg:Binary_Search}.
\end{question}

\begin{solution}
	\begin{algorithm}[H]
		\begin{algorithmic}[1]
			\Require{Array $A$ of size $n$}
			\Ensure{Index of sole $1$ in $A$}
			\Statex
			\Function{Find One}{$A$, $n$}
				\State{$\texttt{min} \gets 1$}
				\State{$\texttt{max} \gets n$}
				\While{$\texttt{min} < \texttt{max}$}
					\State{$\texttt{mid} = \left\lfloor \frac{\texttt{min} + \texttt{max}}{2} \right\rfloor$}
					\If{$\Call{Sum}{A, \texttt{min}, \texttt{mid}} \neq 0$}
						\State{$\texttt{max} = \texttt{min}$}
					\Else
						\State{$\texttt{min} = \texttt{mid} + 1$}
					\EndIf
				\EndWhile
				\State{\Return{$\texttt{min} = \texttt{mid} + 1$}}
			\EndFunction
		\end{algorithmic}
		\caption{Finding switch from $0$s to $1$s in an sorted array of $0$s and $1$s only}
	\end{algorithm}
\end{solution}

\end{document}
