\documentclass[fleqn, a4paper, 11pt, oneside]{amsart}
\usepackage{exsheets}
\usepackage{amsmath, amssymb, amsthm} %standard AMS packages
\usepackage{marginnote} %marginnotes
\usepackage{gensymb} %miscellaneous symbols
\usepackage{commath} %differential symbols
\usepackage{xcolor} %colours
\usepackage{cancel} %cancelling terms
\usepackage[free-standing-units, space-before-unit]{siunitx} %formatting units
\usepackage{tikz, pgfplots} %diagrams
	\usetikzlibrary{calc, hobby, patterns, intersections, decorations.markings}
\usepackage{graphicx} %inserting graphics
\usepackage{hyperref} %hyperlinks
\usepackage{datetime} %date and time
\usepackage{enumerate,enumitem} %numbered lists
\usepackage{float} %inserting floats
\usepackage{circuitikz}[american voltages, american currents] %circuit diagrams
\usepackage{algpseudocode} %algorithms
\usepackage{algorithm} %algorithms

\newcommand\numberthis{\addtocounter{equation}{1}\tag{\theequation}} %adds numbers to specific equations in non-numbered list of equations

\theoremstyle{definition}
\newtheorem{example}{Example}
\newtheorem{definition}{Definition}

\theoremstyle{theorem}
\newtheorem{theorem}{Theorem}

\renewcommand{\algorithmicrequire}{\textbf{Input:}}
\renewcommand{\algorithmicensure}{\textbf{Output:}}

\let\And\relax
\algnewcommand\And{\ \textbf{and}\ }
\algnewcommand\False{\texttt{FALSE}}
\algnewcommand\True{\texttt{TRUE}}

\makeatletter
\@addtoreset{section}{part} %resets section numbers in new part
\makeatother

\SetupExSheets{solution/print = true}

%opening
\title
[
	DS\&A : Assignment 1
]
{
	Data Structures and Algorithms\\
	Assignment 1
}
\author
{
	Aakash Jog\\
	ID : 989323563
}
\date{\formatdate{8}{3}{2016}}

\begin{document}

\maketitle
%\setlength{\mathindent}{0pt}

\begin{question}
	Write a pseudocode for an algorithm that receives as input an array $A$ containing $n$ different numbers sorted in ascending order, and a value $x$ such that
	\begin{equation*}
		A[1] \le x \le A[n]
	\end{equation*}
	but $x$ itself does not appear in the array $A$.
	The algorithm prints pair $(\mathit{lb},\mathit{ub})$ such that
	\begin{enumerate}
		\item Both $\mathit{lb}$ and $\mathit{ub}$ are values which appear in the array $A$.
		\item $\mathit{lb} \le x \le \mathit{ub}$
		\item $\mathit{lb}$ is the closest value to $x$ which is less than or equal to $x$, and $\mathit{ub}$ is the closest value to $x$ which is greater than or equal to $x$.
	\end{enumerate}
	The running time of the algorithm should be $c_1 + c_2 \log(n)$ for some constants $c_1$ and $c_2$.
\end{question}

\begin{solution}
	\begin{algorithm}[H]
		\begin{algorithmic}[1]
			\Require{Array $A$ of size $n$, $x$}
			\Ensure{Lower bound $\mathit{lb}$ of $x$, upper bound $\mathit{ub}$ of $x$}
			\Statex
			\Function{Find Bounds}{$A$, $x$}
				\State{$\mathit{lb} \gets $\Call{Find Lower Bound}{$A$, $x$}}
				\State{$\mathit{ub} \gets $\Call{Find Upper Bound}{$A$, $x$}}
				\State{\Return{$(\mathit{lb},\mathit{ub})$}}
			\EndFunction
			\Statex
			\Function{Find Lower Bound}{$A$, $x$}
				\State{$\texttt{min} \gets 1$}
				\State{$\texttt{max} \gets n$}
				\State{$\texttt{found} = \False$}
				\While{$\texttt{found} = \False$}
					\State{$\texttt{mid} \gets \left\lfloor \frac{\texttt{min} + \texttt{max}}{2} \right\rfloor$}
					\If{$A[\texttt{mid} - 1] < x \And A[\texttt{mid}] > x$}
						\State{$\mathit{lb} \gets A[\texttt{mid} - 1]$}
						\State{$\texttt{found} \gets \True$}
					\ElsIf{$A[\texttt{mid} - 1] < x$}
						\State{$\texttt{max} \gets \texttt{mid}$}
					\Else
						\State{$\texttt{min} \gets \texttt{mid}$}
					\EndIf
				\EndWhile
				\State{\Return{$\mathit{lb}$}}
			\EndFunction
			\Statex
			\Function{Find Upper Bound}{$A$, $x$}
				\State{$\texttt{min} \gets 1$}
				\State{$\texttt{max} \gets n$}
				\State{$\texttt{found} = \False$}
					\State{$\texttt{mid} \gets \left\lfloor \frac{\texttt{min} + \texttt{max}}{2} \right\rfloor$}
				\While{$\texttt{found} = \False$}
					\If{$A[\texttt{mid} - 1] < x \And A[\texttt{mid}] > x$}
						\State{$\mathit{ub} \gets A[\texttt{mid}]$}
						\State{$\texttt{found} \gets \True$}
					\ElsIf{$A[\texttt{mid} - 1] > x$}
						\State{$\texttt{max} \gets \texttt{mid}$}
					\Else
						\State{$\texttt{min} \gets \texttt{mid}$}
					\EndIf
				\EndWhile
				\State{\Return{$\mathit{ub}$}}
			\EndFunction
		\end{algorithmic}
		\caption{}
	\end{algorithm}
\end{solution}

\begin{question}
	Let $A$ be an array that contains $n$ numbers and assume that the values stored in array $A$ are not necessarily different from each other, but still sorted in ascending order.
	Let $x$ be a number.\\
	Write a pseudocode for a procedure whose running time is at most $c_1 + c_2 \log(n)$ for some constants $c_1$ and $c_2$, such that
	\begin{enumerate}
		\item If the number $x$ does not appear in array $A$, the procedure returns \texttt{NOT-FOUND}.
		\item If the number $x$ does appear in array $A$, the procedure returns the index $j$ of the first occurrence of $x$.
	\end{enumerate}
	The procedure can call the \textsc{Binary Search} procedure.
	Give a short explanation.
\end{question}

\begin{solution}
	\begin{algorithm}
		\begin{algorithmic}[1]
			\Require{Array $A$ of size $n$, $x$}
			\Ensure{Index $j$ of first occurrence of $x$}
			\Statex
			\Function{First Occurrence}{$A$, $x$}
				\State{$\texttt{index\_of\_any\_occurrence} \gets$ \Call{Binary Search}{$n$, $A$, $x$}}
				\If{$\texttt{index\_of\_any\_occurrence} = \texttt{NOT-FOUND}$}
					\State{\Return{\texttt{NOT-FOUND}}}
				\EndIf
				\State{$\texttt{min} \gets 1$}
				\State{$\texttt{max} \gets \texttt{index\_of\_any\_occurrence}$}
				\While{$\texttt{min} < \texttt{max}$}
					\State{$\texttt{mid} \gets \left\lfloor \frac{\texttt{min} + \texttt{max}}{2} \right\rfloor$}
					\If{$A[\texttt{mid}] < x$}
						\State{$\texttt{min} \gets \texttt{mid} + 1$}
					\Else
						\State{$\texttt{max} \gets \texttt{mid} + 1$}
					\EndIf
				\EndWhile
				\State{\Return{$\texttt{min}$}}
			\EndFunction
		\end{algorithmic}
		\caption{}
	\end{algorithm}
	The function calls \textsc{Binary Search} to find any occurrence of the required number $x$, and stores this value in \texttt{index\_of\_any\_occurrence}.
	It then takes into consideration the sub-array to the left of this position, including the position itself.
	It `halves' the sub-array, and takes into consideration either the left or the right half depending on whether the central element is $x$ or not.
	It repeats this process, repeatedly `halving' the array, till the length of the sub-array is $1$.
	At this point the only remaining element in the array is the first occurrence of $x$.
	It returns this index.
\end{solution}

\begin{question}
	Let $A$ be an array that contains $n$ numbers.
	The array $A$ is called unimodal if there exists an index $s(A)$, such that
	\begin{itemize}
		\item $1 \le s(A) \le n$.
		\item $A[i] < A[i + 1]$ for each $1 \le i \le s(A)$.
		\item $A[i + 1] < A[i]$ for each $s(A) \le i \le n$.
	\end{itemize}
	\begin{enumerate}
		\item
			What does the value $A[s(A)]$ hold in comparison to the other values of $A$?
			Explain shortly.
		\item
			Write a pseudocode for a procedure that receives as inputs an unimodal array $A$ and its size $n$.
			The procedure returns $s(A)$.\\
			The running time of the procedure should be at most $c_1 + c_2 \log(n)$ for some constants $c_1$ and $c_2$.
			Provide a short explanation of the idea behind the procedure and explain why the running time is the requested one.
	\end{enumerate}
\end{question}

\begin{solution}
	\begin{enumerate}[leftmargin=*]
		\item
			All elements before and including $s(A)$ are in ascending order.
			All elements after and including $s(A)$ are in descending order.
			Therefore, $A[s(A)]$ is the greatest value in the array.
		\item
			~\\
			\begin{algorithm}[H]
				\begin{algorithmic}[1]
					\Require{Array $A$ of size $n$}
					\Ensure{Index $s(A)$ of the mode of $A$}
					\Statex
					\Function{Find Mode}{$A$}
						\State{$\texttt{min} \gets 1$}
						\State{$\texttt{max} \gets n$}
						\State{$\texttt{found} = \False$}
						\While{$\texttt{found} = \False$}
							\State{$\texttt{mid} \gets \left\lfloor \frac{\texttt{min} + \texttt{max}}{2} \right\rfloor$}
							\If{$\texttt{mid} = 1$}
								\If{$A[\texttt{mid}] > A[\texttt{mid} + 1]$}
									\State{$\texttt{mode} = \texttt{mid}$}
									\State{$\texttt{found} \gets \True$}
								\EndIf
							\ElsIf{$\texttt{mid} = n$}
								\If{$A[\texttt{mid} - 1] < A[\texttt{mid}]$}
									\State{$\texttt{mode} = \texttt{mid}$}
									\State{$\texttt{found} \gets \True$}
								\EndIf
							\Else
								\If{$A[\texttt{mid} - 1] < A[\texttt{mid}] > A[\texttt{mid} + 1]$}
									\State{$\texttt{mode} = \texttt{mid}$}
									\State{$\texttt{found} \gets \True$}
								\ElsIf{$A[\texttt{mid} - 1] \le A[\texttt{mid}] \le A[\texttt{mid} + 1]$}
									\State{$\texttt{min} \gets \texttt{mid}$}
								\Else
									\State{$\texttt{max} \gets \texttt{mid}$}
								\EndIf
							\EndIf
						\EndWhile
						\State{\Return{\texttt{mode}}}
					\EndFunction
				\end{algorithmic}
				\caption{}
			\end{algorithm}
			On each execution of the while loop, the algorithm `halves' the array.
			Therefore, the running time is $c_1 + c_2 \log(n)$.
	\end{enumerate}
\end{solution}

\end{document}
